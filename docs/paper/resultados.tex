\section{Resultados}


En la siguiente tabla se detallan los resultados obtenidos para ambas arquitecturas en el conjunto de entrenamiento:

\begin{table}[h]
\begin{tabularx}{.45\textwidth}{||l|X|X|X|X||}
\hline
Embedding & Neuronas & F1 Score FNN & F1 Score Fish                  \\ \hline\hline
128       &    48    & 0.723873     & 0.726558 \cellcolor{green!20}  \\ \hline
128       &    64    & 0.734102     & 0.720681 \cellcolor{red!30}    \\ \hline
128       &    88    & 0.707602     & 0.715352 \cellcolor{green!20}  \\ \hline               
256       &    64    & 0.720902     & 0.717698 \cellcolor{red!30}    \\ \hline
256       &    128   & 0.719521     & 0.725211 \cellcolor{green!20}  \\ \hline
256       &    192   & 0.720952     & 0.714286 \cellcolor{red!30}    \\ \hline
512       &    64    & 0.669626     & 0.724931 \cellcolor{green!20}  \\ \hline
512       &    128   & 0.709508     & 0.722067 \cellcolor{green!20}  \\ \hline
512       &    256   & 0.717080     & 0.720137 \cellcolor{green!20}  \\ \hline
\end{tabularx}
\caption{\label{fig:resultados-fish} \textit{F1-score} en el conjunto de entrenamiento para ambas arquitecturas}
\end{table}

Solo para tres configuraciones, la arquitectura que utiliza un \textit{Autoencoder} tiene un mejor \textit{F1-Score} que la arquitectura propuesta \textit{Fish}.

El uso de \textit{F1 Score} es necesario pues, tal como se mencionó en la sección 2.3, el conjunto de datos de prueba mantiene la proporción del $10\%$ de \textit{tweets} tóxicos del conjunto de datos original. Mientras que métricas como el \textit{accuracy} ocultan la verdadera capacidad de generalización de modelos cuando se miden en un conjunto de datos desbalanceado, \textit{F1 Score} informa una media armónica entre la \textit{precision} y el \textit{recall}: dos métricas que miden con mayor robustez el desempeño en este tipo de conjuntos de datos.