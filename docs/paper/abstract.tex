\begin{abstract}
La familia de modelos de \textit{Autoencoder} ha demostrado ser eficiente en la construcción de codificadores que permiten obtener una representación de menor dimensionalidad que capturan y codifican aquellos atributos ocultos que permiten posteriormente reconstruir la entrada. Sin embargo, estas codificaciones no consideran la tarea final en la cual estas representaciones serán utilizadas pues la reducción se aborda como un problema \textit{self-supervised} independiente.  En la presente investigación se propone \textbf{Fish}: una simple y novedosa arquitectura neuronal que permite construir un codificador $\textit{ad-hoc}$ a la tarea supervisada.
\end{abstract}

% Obtener representaciones de menor dimensionalidad que capturen y codifiquen aquellos atributos claves y ocultos de los datos ha sido una constante búsqueda en el área de Aprendizaje de Máquinas. Múltiples algoritmos solucionan este problema, pero ninguno de ellos considerando la tarea final en la cual estas representaciones son utilizadas. 